\documentclass[12pt,a4paper]{article}

% -------------------------------------------------------
% PACCHETTI DI BASE
% -------------------------------------------------------
\usepackage[utf8]{inputenc}      % Per lettere accentate
\usepackage[T1]{fontenc}
\usepackage[italian]{babel}      % Lingua italiana
\usepackage{lmodern}             % Font migliore
\usepackage{setspace}            % Gestione spaziatura

% -------------------------------------------------------
% MATEMATICA
% -------------------------------------------------------
\usepackage{amsmath, amssymb, amsthm}   % Matematica standard
\usepackage{mathtools}                  % Miglioramenti AMS
\usepackage{physics}                    % Comandi per derivate, vettori, ecc.
\usepackage{bm}                         % Per vettori e simboli in grassetto

% -------------------------------------------------------
% GRAFICA E TABELLE
% -------------------------------------------------------
\usepackage{graphicx}                   % Inserimento immagini
\usepackage{caption}
\usepackage{subcaption}
\usepackage{booktabs}                   % Tabelle professionali
\usepackage{array}                      % Miglior controllo tabelle
\usepackage{float}                      % Figure dove vuoi tu

% -------------------------------------------------------
% CODICE (opzionale)
% -------------------------------------------------------
\usepackage{listings}                   
\usepackage{xcolor}                     
\lstset{
    basicstyle=\ttfamily\small,
    numbers=left,
    numberstyle=\tiny,
    frame=single,
    breaklines=true
}

% -------------------------------------------------------
% IMPOSTAZIONI PAGINA
% -------------------------------------------------------
\usepackage[a4paper, margin=2.5cm]{geometry}
\setstretch{1.15}                       % Interlinea

% -------------------------------------------------------
% COMANDI PERSONALIZZATI
% -------------------------------------------------------
\newcommand{\R}{\mathbb{R}}             % Shortcut per numeri reali
\newcommand{\vect}[1]{\mathbf{#1}}      % Vettori in grassetto
\newcommand{\absd}[1]{\left|#1\right|}  % Valore assoluto

% -------------------------------------------------------
% FINE PREAMBOLO
% -------------------------------------------------------


% -------------------------------------------------------
% Titolo e autore
% -------------------------------------------------------
\title{Conversione di un'EDO del Secondo Ordine in un Sistema di Primo Ordine}
\author{Gabriele Cembalo}


\begin{document}

\maketitle
\tableofcontents

\section{Obiettivo}

Dato un problema descritto da un'equazione differenziale ordinaria (EDO) di \textbf{secondo ordine}, vogliamo trasformarlo in un \textbf{sistema equivalente di equazioni del primo ordine}. Questo è utile per l'analisi qualitativa e soprattutto per la risoluzione numerica.


\section{Forme iniziali}

\textbf{Caso scalare}:
\begin{equation}
    y'' = f(t, y, y')
\end{equation}

\noindent\textbf{Caso vettoriale}:
\begin{align}
    &\mathbf{y}'' = \mathbf{F}(t, \mathbf{y}, \mathbf{y}') \\
    &\mathbf{y}, \mathbf{y}' \in \mathbb{R}^n
\end{align}


\subsection{Procedura di conversione}


\subsubsection{Introduzione delle variabili ausiliarie}

Definiamo la derivata prima come una nuova variabile:
\begin{equation}
    v = y' \qquad\text{(caso scalare)}
\end{equation}
\begin{equation}
    \mathbf{v} = \mathbf{y}'
    \qquad\text{(caso vettoriale)}
\end{equation}

\subsubsection{Riscrittura del sistema del primo ordine}

\paragraph{Caso scalare:}
\begin{equation}
    \begin{cases}
        y' = v \\
        v' = f(t, y, v)
    \end{cases}
\end{equation}

\paragraph{Caso vettoriale:}
\begin{equation}
    \begin{cases}
        \mathbf{y}' = \mathbf{v} \\[6pt]
        \mathbf{v}' = \mathbf{F}(t, \mathbf{y}, \mathbf{v})
    \end{cases}
\end{equation}

\subsubsection{Forma compatta con vettore di stato}
Definiamo il vettore di stato:
\begin{equation}
    \mathbf{x} = \begin{pmatrix}
        \mathbf{y} \\
        \mathbf{v}
    \end{pmatrix} \in \mathbb{R}^{2n}
\end{equation}

Allora il sistema diventa:
\begin{equation}
    \mathbf{x}' = \mathbf{G}(t, \mathbf{x})
\end{equation}


\subsubsection{Condizioni iniziali}

Per un'EDO di secondo ordine, servono:
\begin{equation}
    y(t_0) = y_0, \qquad y'(t_0) = v_0
\end{equation}

In forma di stato:

\begin{equation}
    \mathbf{x}(t_0) = \begin{pmatrix}
        y_0 \\
        v_0
    \end{pmatrix}
\end{equation}


\section{Associazione tra indici del vettore di stato e RHS}

Quando si implementa numericamente un sistema del primo ordine, si usa spesso la notazione:
\begin{equation}
    \mathbf{Y} = \begin{pmatrix}
        Y[0] \\
        Y[1] \\
        \vdots \\
        Y[n]
    \end{pmatrix} , \qquad \mathbf{R} = \begin{pmatrix}
        R[0] \\
        R[1] \\
        \vdots \\
        R[n]
    \end{pmatrix}
\end{equation}
dove $\mathbf{R}$ rappresenta la \textit{Right-Hand-Side function}, ovvero $\mathbf{R} = \dfrac{d\mathbf{Y}}{dt}$.


\subsection{Regola fondamentale}

\begin{equation}
    \boxed{R[i] = \dfrac{d}{dt}Y[i]}
\end{equation}

Questo significa che l'\textbf{ordine degli elementi in $\mathbf{R}$ deve rispettare esattamente l'ordine delle variabili in $\mathbf{Y}$}.


\subsection{Esempio (caso scalare)}

Dato:
\begin{equation}
    y'' = f(t,y,y')
\end{equation}

definiamo:
\begin{equation}
    Y[0] = y, \qquad Y[1] = y'
\end{equation}
Allora le equazioni diventano:
\begin{equation}
    \begin{cases}
        R[0] = \dfrac{d}{dt}Y[0] = Y[1] \\
        R[1] = \dfrac{d}{dt}Y[1] = f(t,Y[0],Y[1])
    \end{cases}
\end{equation}


\subsection{Osservazioni fondamentali}
\begin{itemize}
    \item Non esiste un ordine “giusto” universale per $Y[i]$, \textbf{ma} qualsiasi ordine si scelga deve essere mantenuto coerente anche in $\mathbf{R}$.
    \item Se si cambia l'ordine delle variabili in $\mathbf{Y}$, si deve cambiare anche l'ordine delle corrispondenti equazioni in $\mathbf{R}$.
    \item L'associazione è sempre diretta: $Y[i] \longleftrightarrow R[i]$.
\end{itemize}


\subsection{Schema riassuntivo}
\begin{equation}
    \boxed{
        \mathbf{Y} = \begin{pmatrix}
            y \\
            y'
        \end{pmatrix} 
    \quad\Rightarrow\quad
    \mathbf{R} = \begin{pmatrix}
        y' \\
        f(t,y,y')
    \end{pmatrix}
    }
\end{equation}


\section{Esempi applicativi}

\subsection{Oscillatore armonico semplice}
\begin{equation}
    y'' + \omega^2 y = 0
\end{equation}
\begin{equation}
    \begin{cases}
        y' = v \\
        v' = -\omega^2 y
    \end{cases}
\end{equation}


\subsection{Pendolo semplice (non linearizzato)}
\begin{equation}
    \theta'' + \dfrac{g}{L} \sin \theta = 0
\end{equation}
\begin{equation}
    \begin{cases}
        \theta' = \omega \\
        \omega' = -\dfrac{g}{L} \sin\theta
    \end{cases}
\end{equation}


\subsection{Equazione del moto gravitazionale}
\begin{equation}
    \frac{d^2 \mathbf{x}}{d t^2} = -\frac{GM}{|\mathbf{x}|^3} \mathbf{x}
\end{equation}
\begin{equation}
    \begin{cases}
        \mathbf{x}' = \mathbf{v} \\[4pt]
        \mathbf{v}' = -\dfrac{GM}{|\mathbf{x}|^3} \mathbf{x}
    \end{cases}
\end{equation}


\section{Ricetta riassuntiva}
\begin{enumerate}
    \item Data un'equazione del tipo $y'' = f(t, y, y')$, introduci $v = y'$.
    \item Scrivi il sistema:
    \begin{equation}
        \begin{cases}
            y' = v \\
            v' = f(t, y, v)
        \end{cases}
    \end{equation}
    \item Definisci il vettore di stato $\mathbf{Y} = (y, v)^T$.
    \item Costruisci la RHS rispettando la corrispondenza:
    \begin{equation}
        R[i] = \dfrac{d}{dt}Y[i]
    \end{equation}
    \item Specifica le condizioni iniziali $y(t_0)$ e $v(t_0)$.
\end{enumerate}


\subsection{Utilità pratica}
Questa procedura consente di applicare direttamente metodi numerici di integrazione (Euler, Runge--Kutta, metodi impliciti) e facilita lo studio qualitativo del sistema, come stabilità, attrattori o linearizzazione.

\end{document}